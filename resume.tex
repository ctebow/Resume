\documentclass[10pt]{article}

\usepackage[margin=0.7in]{geometry}
\usepackage{enumitem}
\usepackage{hyperref}
\usepackage{titlesec}
\usepackage{setspace}

\setlength{\parindent}{0pt}
\setlist[itemize]{noitemsep, topsep=0pt}
\pagenumbering{gobble}

\titleformat{\section}{\bfseries\large}{}{0em}{}[\titlerule]
\titlespacing{\section}{0pt}{6pt}{4pt}

\begin{document}

\begin{center}
{\LARGE \textbf{Caden Tebow}} \\[4pt]
\href{mailto:ctebow@uchicago.edu}{ctebow@uchicago.edu} \;|\; (612) 695-8210 \;|\;
\href{https://github.com/ctebow}{github.com/ctebow} \;|\; LinkedIn
\end{center}

\vspace{-6pt}

\section*{Education}

\textbf{The University of Chicago} \hfill Chicago, IL \\
Bachelor of Science in Physics and Computer Science (GPA: 3.92/4.00) \hfill Expected June 2028 \\
Relevant Coursework: DSA, Differential Equations, Systems Programming, Quantum Mechanics \\
Awards: Dean’s List 2024-2025, U.S. Presidential Scholar Semi-Finalist \\
Activities: Kappa Theta Pi (Technology Fraternity), PULSE-A, Tech Team, Chamber Orchestra, Physics Society

\vspace{2pt}

\section*{Skills}

Programming Languages: Python, C, C++, JavaScript \\
Libraries: NumPy, SciPy, OpenCV, MatPlotLib, PyTorch, FastAPI, React.js \\
Developer Tools: Git, Linux, Docker, Valgrind, GDB, Criterion, Firmware Development (Teensy, Arduino)


\section*{Experience}

\textbf{Kavli Institute for Cosmological Physics: UChicago Vieregg Lab} \hfill Chicago, IL \\
Undergraduate Researcher \hfill January 2025 – Present
\begin{itemize}
\item Engineer custom firmware for radar sensors, integrating control via a Teensy microcontroller and Python automation scripts, validating the performance of radar sensors and creating faster testing workflows while achieving 1mm accuracy.
\item Design end-to-end custom PCB hardware, creating a KiCad-based breakout board for a precision distance sensor, selecting components, building full schematics, running electrical simulations and design checks, and managing manufacturing.
\end{itemize}

\textbf{PULSE-A CubeSat - NASA Undergraduate Satellite} \hfill Chicago, IL \\
Payload Engineer \hfill September 2025 – Present
\begin{itemize}
\item Develop finite-state-machine spiral-search algorithms for a 3U CubeSat’s acquisition system, enabling autonomous target detection for a 2027 mission demonstrating laser communications via circular-polarized shift-keying (CPolSK).
\item Implement a closed-loop fine-pointing control algorithm that computes micro-adjustments from quad-photodiode voltage differentials, enhancing tracking stability and supporting reliable high-bandwidth optical downlink performance.
\item Benchmark, test, and deploy a MEMS Fine Steering Mirror (FSM) with custom C++ firmware to $\sim$3\% error from documented benchmarks while in an ESD and cleanroom setting.
\end{itemize}

\textbf{Connections for Abused Women and their Children} \hfill Chicago, IL \\
Software Developer \hfill August 2025 – December 2025
\begin{itemize}
\item Re-designed JS/Flask website of local non-profit organization for clarity and data privacy, with over 200,000 total impressions.
\end{itemize}

\section*{Projects}

\textbf{Drawn Circuit Identifier}:
\href{https://github.com/ctebow/breadMaker-backend}{github.com/ctebow/breadMaker-backend} \hfill June 2025
\begin{itemize}
\item Developed a machine learning model that was trained on 10,000+ images using PyTorch and OpenCV for automatic hand-drawn circuit element uploads, streamlining the process to be 2x faster.
\item Tuned model over multiple epochs using a local GPU and manually bounded classes, with validation on a 300-image test set achieving an 85\% mAP@50 with $>$80\% confidence across 30+ circuit classes.
\end{itemize}

\textbf{Retrieval-Augmented Generation Pipeline}:
\href{https://github.com/ctebow/Local-RAG-with-Gemini}{github.com/ctebow/Local-RAG-with-Gemini} \hfill August 2025
\begin{itemize}
\item Engineered a retrieval-augmented generation pipeline using vector embeddings with chunking and top-k search, enabling for automatic context generation when prompting an LLM, resulting in code-base aware development assistance.
\item Optimized for constrained local hardware to support 2,000+ token responses within a 4,000-token context window, allowing for $<$20 second response times and saving 15 seconds per LLM prompt.
\end{itemize}

\textbf{Circuit Editing Website}:
\href{https://bread-maker-frontend-git-main-cadens-projects-d48b21c6.vercel.app/}{Vercel Deployment} \hfill June 2025
\begin{itemize}
\item Built and deployed circuit drawing website on Vercel using React.js, Node.js, and FastAPI that uses wire-drawing and connection heuristics to allow for node identification, circuit connections generation, and exporting into .net file format.
\end{itemize}

\end{document}
